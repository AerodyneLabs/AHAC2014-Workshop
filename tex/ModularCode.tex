\section{Writing Modular Code}
\label{sec:modular_code}

\subsection{Analog Input}
\label{sub:analog_input}
\begin{frame}[fragile]
	\frametitle{Analog Input}
	AnalogIn operates almost identically to DigitalIn:
	\begin{lstlisting}[numbers=none]
		AnalogIn in(PinName pin);
	\end{lstlisting}
	\vfill
	You can access the value by treating it like a variable or with a function:
	\begin{lstlisting}[]
		float value = in;
		float value = in.read();
		unsigned short value = in.read_u16();
	\end{lstlisting}
	1 and 2 are identical and return a float in the range [0.0 -- 1.0].\\
	3 returns an unsigned short in the range [0x0000 -- 0xFFFF].
\end{frame}

\begin{frame}
	\frametitle{Using Potentiometers}
	\begin{columns}[T]
		\begin{column}{0.5\textwidth}
			A potentiometer functions as an adjustable voltage divider:
			\begin{itemize}
				\item One leg connects to 3.3v
				\item One leg connects to Gnd
				\item Wiper connects to an analog pin
			\end{itemize}
			The wiper varies between 0 and 3.3v as you turn the knob.
		\end{column}
		\begin{column}{0.5\textwidth}
			%TODO potentiometer schematic
			%TODO hookup picture
		\end{column}
	\end{columns}
\end{frame}

\begin{frame}[fragile]
	\frametitle{Analog Input Demo}
	\begin{columns}[T]
		\begin{column}{0.5\textwidth}
			Create a new program:
			\begin{itemize}
				\item AnalogIn connected to potentiometer
				\item Serial port connected to USB
				\item Print the analog value to serial
				\item Try using format specifiers with printf
			\end{itemize}
		\end{column}
		\pause
		\begin{column}{0.5\textwidth}
			\lstinputlisting[title=Analog.cpp]{code/modular_code/analog.cpp}
		\end{column}
	\end{columns}
\end{frame}

\subsection{PWM Output}
\label{sub:pwm_output}
\begin{frame}
	\frametitle{PWM Output}
	\begin{columns}[c]
		\begin{column}{0.5\textwidth}
			Pulse Width Modulation encodes a signal by varying the width of a series of pulses
		\end{column}
		\begin{column}{0.5\textwidth}
			\includegraphics[width=\linewidth]{pwm}
		\end{column}
	\end{columns}
	\ccbysa{http://commons.wikimedia.org/wiki/File:Pwm.png}{CyrilB}
\end{frame}

\subsection{Classes}
\label{sub:classes}
