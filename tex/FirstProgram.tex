\section{Your First Program}
\label{sec:first_program}

\subsection{While Loops}
\label{sub:while_loops}
\begin{frame}[fragile]
	\frametitle{While Loops}
	\begin{lstlisting}[keepspaces=true]
		while(conditional) {
		  // Code to execute when true
		}
	\end{lstlisting}
	While loops executes the code contained within them while their conditional statement is true.
	\begin{block}{Example}
		A main function should (almost) never exit:
		\begin{lstlisting}[keepspaces=true]
			int main() {
			  while(true) {
			    // Main loop code, runs forever
			  }
			}
		\end{lstlisting}
	\end{block}
\end{frame}

\begin{frame}
	\frametitle{Your Program}
	\begin{columns}[c]
		\begin{column}{0.5\textwidth}
			\lstinputlisting[caption=main.cpp]{code/first_program/1.cpp}
		\end{column}
		\begin{column}{0.5\textwidth}
			\begin{itemize}
				\item Add an infinite while loop to your main function to prevent your program from ending.
			\end{itemize}
		\end{column}
	\end{columns}
\end{frame}

\subsection{Digital Output}
\label{sub:digital_output}
\begin{frame}[fragile]
	\frametitle{Digital Output}
	DigitalOut constructor:
	\begin{lstlisting}[numbers=none]
		DigitalOut(PinName pin)
	\end{lstlisting}
	It creates an object attached to the given pin. Anytime you see PinName, use a name from the images on the next slide.
	
	You can assign a value to the object using the equals sign. 1 turns the pin on while a 0 turns the pin off.
	\begin{block}{Example}
		Attach a DigitalOut to the LED1 pin on the Nucleo and turn it on:
		\begin{lstlisting}[numbers=none]
			DigitalOut led(LED1);
			led = 1;
		\end{lstlisting}
	\end{block}
\end{frame}

\begin{frame}
	\frametitle{Nucleo Pin Names}
	\begin{columns}[c]
		\begin{column}{0.5\textwidth}
			\includegraphics[width=\linewidth]{arduino_pins}
		\end{column}
		\begin{column}{0.5\textwidth}
			\includegraphics[width=\linewidth]{morpho_pins}
		\end{column}
	\end{columns}
	\begin{block}{Warning}
		You can only use the labels in blue and green!
	\end{block}
	\begin{center}
		\small Full size versions are available at \url{https://mbed.org/platforms/ST-Nucleo-F401RE/}
	\end{center}
\end{frame}

\begin{frame}
	\frametitle{Your Program}
	\begin{columns}[c]
		\begin{column}{0.55\textwidth}
			\lstinputlisting[caption=main.cpp]{code/first_program/2.cpp}
		\end{column}
		\begin{column}{0.45\textwidth}
			\begin{itemize}
				\item Declare a global DigitalOut object
				\item Turn the output on and off in your main loop
			\end{itemize}
			\begin{block}{Note}
				The LED won't seem to be flashing, but it actually is at about 42 MHz, much faster than your eye.
			\end{block}
		\end{column}
	\end{columns}
\end{frame}

\subsection{Waiting}
\label{sub:waiting}
\begin{frame}[fragile]
	\frametitle{Waiting}
	There are three statements that can slow down execution:
	\begin{lstlisting}[numbers=none]
		void wait(float s);
		void wait_ms(int ms);
		void wait_us(int us);
	\end{lstlisting}
	All three will pause execution for the amount of time specified. Use these statements any time you need a controlled delay.
	\begin{block}{Notice}
		Wait and other block statements can have some unintended side effects. This will be demonstrated later.
	\end{block}
\end{frame}

\begin{frame}
	\frametitle{Your Program}
	\begin{columns}[c]
		\begin{column}{0.6\textwidth}
			\lstinputlisting[caption=main.cpp]{code/first_program/3.cpp}
		\end{column}
		\begin{column}{0.4\textwidth}
			\begin{itemize}
				\item Add a wait statement after each write to your output
				\item You should now be able to see your LED flashing
				\item Try making your own patterns!
			\end{itemize}
		\end{column}
	\end{columns}
\end{frame}